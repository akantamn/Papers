
%% bare_conf.tex
%% V1.3
%% 2007/01/11
%% by Michael Shell
%% See:
%% http://www.michaelshell.org/
%% for current contact information.
%%
%% This is a skeleton file demonstrating the use of IEEEtran.cls
%% (requires IEEEtran.cls version 1.7 or later) with an IEEE conference paper.
%%
%% Support sites:
%% http://www.michaelshell.org/tex/ieeetran/
%% http://www.ctan.org/tex-archive/macros/latex/contrib/IEEEtran/
%% and
%% http://www.ieee.org/

%%*************************************************************************
%% Legal Notice:
%% This code is offered as-is without any warranty either expressed or
%% implied; without even the implied warranty of MERCHANTABILITY or
%% FITNESS FOR A PARTICULAR PURPOSE! 
%% User assumes all risk.
%% In no event shall IEEE or any contributor to this code be liable for
%% any damages or losses, including, but not limited to, incidental,
%% consequential, or any other damages, resulting from the use or misuse
%% of any information contained here.
%%
%% All comments are the opinions of their respective authors and are not
%% necessarily endorsed by the IEEE.
%%
%% This work is distributed under the LaTeX Project Public License (LPPL)
%% ( http://www.latex-project.org/ ) version 1.3, and may be freely used,
%% distributed and modified. A copy of the LPPL, version 1.3, is included
%% in the base LaTeX documentation of all distributions of LaTeX released
%% 2003/12/01 or later.
%% Retain all contribution notices and credits.
%% ** Modified files should be clearly indicated as such, including  **
%% ** renaming them and changing author support contact information. **
%%
%% File list of work: IEEEtran.cls, IEEEtran_HOWTO.pdf, bare_adv.tex,
%%                    bare_conf.tex, bare_jrnl.tex, bare_jrnl_compsoc.tex
%%*************************************************************************

% *** Authors should verify (and, if needed, correct) their LaTeX system  ***
% *** with the testflow diagnostic prior to trusting their LaTeX platform ***
% *** with production work. IEEE's font choices can trigger bugs that do  ***
% *** not appear when using other class files.                            ***
% The testflow support page is at:
% http://www.michaelshell.org/tex/testflow/



% Note that the a4paper option is mainly intended so that authors in
% countries using A4 can easily print to A4 and see how their papers will
% look in print - the typesetting of the document will not typically be
% affected with changes in paper size (but the bottom and side margins will).
% Use the testflow package mentioned above to verify correct handling of
% both paper sizes by the user's LaTeX system.
%
% Also note that the "draftcls" or "draftclsnofoot", not "draft", option
% should be used if it is desired that the figures are to be displayed in
% draft mode.
%
\documentclass[12pt,draftcls,onecolumn]{IEEEtran}
% Add the compsoc option for Computer Society conferences.
%
% If IEEEtran.cls has not been installed into the LaTeX system files,
% manually specify the path to it like:
% \documentclass[conference]{../sty/IEEEtran}





% Some very useful LaTeX packages include:
% (uncomment the ones you want to load)


% *** MISC UTILITY PACKAGES ***
%
%\usepackage{ifpdf}
% Heiko Oberdiek's ifpdf.sty is very useful if you need conditional
% compilation based on whether the output is pdf or dvi.
% usage:
% \ifpdf
%   % pdf code
% \else
%   % dvi code
% \fi
% The latest version of ifpdf.sty can be obtained from:
% http://www.ctan.org/tex-archive/macros/latex/contrib/oberdiek/
% Also, note that IEEEtran.cls V1.7 and later provides a builtin
% \ifCLASSINFOpdf conditional that works the same way.
% When switching from latex to pdflatex and vice-versa, the compiler may
% have to be run twice to clear warning/error messages.






% *** CITATION PACKAGES ***
%
%\usepackage{cite}
% cite.sty was written by Donald Arseneau
% V1.6 and later of IEEEtran pre-defines the format of the cite.sty package
% \cite{} output to follow that of IEEE. Loading the cite package will
% result in citation numbers being automatically sorted and properly
% "compressed/ranged". e.g., [1], [9], [2], [7], [5], [6] without using
% cite.sty will become [1], [2], [5]--[7], [9] using cite.sty. cite.sty's
% \cite will automatically add leading space, if needed. Use cite.sty's
% noadjust option (cite.sty V3.8 and later) if you want to turn this off.
% cite.sty is already installed on most LaTeX systems. Be sure and use
% version 4.0 (2003-05-27) and later if using hyperref.sty. cite.sty does
% not currently provide for hyperlinked citations.
% The latest version can be obtained at:
% http://www.ctan.org/tex-archive/macros/latex/contrib/cite/
% The documentation is contained in the cite.sty file itself.






% *** GRAPHICS RELATED PACKAGES ***
%
\ifCLASSINFOpdf
  \usepackage[pdftex]{graphicx}
  % declare the path(s) where your graphic files are
  % \graphicspath{{../pdf/}{../jpeg/}}
  % and their extensions so you won't have to specify these with
  % every instance of \includegraphics
  % \DeclareGraphicsExtensions{.pdf,.jpeg,.png}
\else
  % or other class option (dvipsone, dvipdf, if not using dvips). graphicx
  % will default to the driver specified in the system graphics.cfg if no
  % driver is specified.
  % \usepackage[dvips]{graphicx}
  % declare the path(s) where your graphic files are
  % \graphicspath{{../eps/}}
  % and their extensions so you won't have to specify these with
  % every instance of \includegraphics
  % \DeclareGraphicsExtensions{.eps}
\fi
% graphicx was written by David Carlisle and Sebastian Rahtz. It is
% required if you want graphics, photos, etc. graphicx.sty is already
% installed on most LaTeX systems. The latest version and documentation can
% be obtained at: 
% http://www.ctan.org/tex-archive/macros/latex/required/graphics/
% Another good source of documentation is "Using Imported Graphics in
% LaTeX2e" by Keith Reckdahl which can be found as epslatex.ps or
% epslatex.pdf at: http://www.ctan.org/tex-archive/info/
%
% latex, and pdflatex in dvi mode, support graphics in encapsulated
% postscript (.eps) format. pdflatex in pdf mode supports graphics
% in .pdf, .jpeg, .png and .mps (metapost) formats. Users should ensure
% that all non-photo figures use a vector format (.eps, .pdf, .mps) and
% not a bitmapped formats (.jpeg, .png). IEEE frowns on bitmapped formats
% which can result in "jaggedy"/blurry rendering of lines and letters as
% well as large increases in file sizes.
%
% You can find documentation about the pdfTeX application at:
% http://www.tug.org/applications/pdftex





% *** MATH PACKAGES ***
%
%\usepackage[cmex10]{amsmath}
% A popular package from the American Mathematical Society that provides
% many useful and powerful commands for dealing with mathematics. If using
% it, be sure to load this package with the cmex10 option to ensure that
% only type 1 fonts will utilized at all point sizes. Without this option,
% it is possible that some math symbols, particularly those within
% footnotes, will be rendered in bitmap form which will result in a
% document that can not be IEEE Xplore compliant!
%
% Also, note that the amsmath package sets \interdisplaylinepenalty to 10000
% thus preventing page breaks from occurring within multiline equations. Use:
%\interdisplaylinepenalty=2500
% after loading amsmath to restore such page breaks as IEEEtran.cls normally
% does. amsmath.sty is already installed on most LaTeX systems. The latest
% version and documentation can be obtained at:
% http://www.ctan.org/tex-archive/macros/latex/required/amslatex/math/





% *** SPECIALIZED LIST PACKAGES ***
%
%\usepackage{algorithmic}
% algorithmic.sty was written by Peter Williams and Rogerio Brito.
% This package provides an algorithmic environment fo describing algorithms.
% You can use the algorithmic environment in-text or within a figure
% environment to provide for a floating algorithm. Do NOT use the algorithm
% floating environment provided by algorithm.sty (by the same authors) or
% algorithm2e.sty (by Christophe Fiorio) as IEEE does not use dedicated
% algorithm float types and packages that provide these will not provide
% correct IEEE style captions. The latest version and documentation of
% algorithmic.sty can be obtained at:
% http://www.ctan.org/tex-archive/macros/latex/contrib/algorithms/
% There is also a support site at:
% http://algorithms.berlios.de/index.html
% Also of interest may be the (relatively newer and more customizable)
% algorithmicx.sty package by Szasz Janos:
% http://www.ctan.org/tex-archive/macros/latex/contrib/algorithmicx/




% *** ALIGNMENT PACKAGES ***
%
%\usepackage{array}
% Frank Mittelbach's and David Carlisle's array.sty patches and improves
% the standard LaTeX2e array and tabular environments to provide better
% appearance and additional user controls. As the default LaTeX2e table
% generation code is lacking to the point of almost being broken with
% respect to the quality of the end results, all users are strongly
% advised to use an enhanced (at the very least that provided by array.sty)
% set of table tools. array.sty is already installed on most systems. The
% latest version and documentation can be obtained at:
% http://www.ctan.org/tex-archive/macros/latex/required/tools/


%\usepackage{mdwmath}
%\usepackage{mdwtab}
% Also highly recommended is Mark Wooding's extremely powerful MDW tools,
% especially mdwmath.sty and mdwtab.sty which are used to format equations
% and tables, respectively. The MDWtools set is already installed on most
% LaTeX systems. The lastest version and documentation is available at:
% http://www.ctan.org/tex-archive/macros/latex/contrib/mdwtools/


% IEEEtran contains the IEEEeqnarray family of commands that can be used to
% generate multiline equations as well as matrices, tables, etc., of high
% quality.


%\usepackage{eqparbox}
% Also of notable interest is Scott Pakin's eqparbox package for creating
% (automatically sized) equal width boxes - aka "natural width parboxes".
% Available at:
% http://www.ctan.org/tex-archive/macros/latex/contrib/eqparbox/





% *** SUBFIGURE PACKAGES ***
%\usepackage[tight,footnotesize]{subfigure}
% subfigure.sty was written by Steven Douglas Cochran. This package makes it
% easy to put subfigures in your figures. e.g., "Figure 1a and 1b". For IEEE
% work, it is a good idea to load it with the tight package option to reduce
% the amount of white space around the subfigures. subfigure.sty is already
% installed on most LaTeX systems. The latest version and documentation can
% be obtained at:
% http://www.ctan.org/tex-archive/obsolete/macros/latex/contrib/subfigure/
% subfigure.sty has been superceeded by subfig.sty.



%\usepackage[caption=false]{caption}
%\usepackage[font=footnotesize]{subfig}
% subfig.sty, also written by Steven Douglas Cochran, is the modern
% replacement for subfigure.sty. However, subfig.sty requires and
% automatically loads Axel Sommerfeldt's caption.sty which will override
% IEEEtran.cls handling of captions and this will result in nonIEEE style
% figure/table captions. To prevent this problem, be sure and preload
% caption.sty with its "caption=false" package option. This is will preserve
% IEEEtran.cls handing of captions. Version 1.3 (2005/06/28) and later 
% (recommended due to many improvements over 1.2) of subfig.sty supports
% the caption=false option directly:
%\usepackage[caption=false,font=footnotesize]{subfig}
%
% The latest version and documentation can be obtained at:
% http://www.ctan.org/tex-archive/macros/latex/contrib/subfig/
% The latest version and documentation of caption.sty can be obtained at:
% http://www.ctan.org/tex-archive/macros/latex/contrib/caption/




% *** FLOAT PACKAGES ***
%
%\usepackage{fixltx2e}
% fixltx2e, the successor to the earlier fix2col.sty, was written by
% Frank Mittelbach and David Carlisle. This package corrects a few problems
% in the LaTeX2e kernel, the most notable of which is that in current
% LaTeX2e releases, the ordering of single and double column floats is not
% guaranteed to be preserved. Thus, an unpatched LaTeX2e can allow a
% single column figure to be placed prior to an earlier double column
% figure. The latest version and documentation can be found at:
% http://www.ctan.org/tex-archive/macros/latex/base/



%\usepackage{stfloats}
% stfloats.sty was written by Sigitas Tolusis. This package gives LaTeX2e
% the ability to do double column floats at the bottom of the page as well
% as the top. (e.g., "\begin{figure*}[!b]" is not normally possible in
% LaTeX2e). It also provides a command:
%\fnbelowfloat
% to enable the placement of footnotes below bottom floats (the standard
% LaTeX2e kernel puts them above bottom floats). This is an invasive package
% which rewrites many portions of the LaTeX2e float routines. It may not work
% with other packages that modify the LaTeX2e float routines. The latest
% version and documentation can be obtained at:
% http://www.ctan.org/tex-archive/macros/latex/contrib/sttools/
% Documentation is contained in the stfloats.sty comments as well as in the
% presfull.pdf file. Do not use the stfloats baselinefloat ability as IEEE
% does not allow \baselineskip to stretch. Authors submitting work to the
% IEEE should note that IEEE rarely uses double column equations and
% that authors should try to avoid such use. Do not be tempted to use the
% cuted.sty or midfloat.sty packages (also by Sigitas Tolusis) as IEEE does
% not format its papers in such ways.





% *** PDF, URL AND HYPERLINK PACKAGES ***
%
%\usepackage{url}
% url.sty was written by Donald Arseneau. It provides better support for
% handling and breaking URLs. url.sty is already installed on most LaTeX
% systems. The latest version can be obtained at:
% http://www.ctan.org/tex-archive/macros/latex/contrib/misc/
% Read the url.sty source comments for usage information. Basically,
% \url{my_url_here}.





% *** Do not adjust lengths that control margins, column widths, etc. ***
% *** Do not use packages that alter fonts (such as pslatex).         ***
% There should be no need to do such things with IEEEtran.cls V1.6 and later.
% (Unless specifically asked to do so by the journal or conference you plan
% to submit to, of course. )


% correct bad hyphenation here
\hyphenation{op-tical net-works semi-conduc-tor}


\begin{document}
%
% paper title
% can use linebreaks \\ within to get better formatting as desired
\title{Addressing the Usability Concerns of the Elderly: A Proposal}


% author names and affiliations
% use a multiple column layout for up to three different
% affiliations
\author{\IEEEauthorblockN{Abhilash Kantamneni}
\IEEEauthorblockA{Department of Computer Science\\
Michigan Technological University\\
}
}

% conference papers do not typically use \thanks and this command
% is locked out in conference mode. If really needed, such as for
% the acknowledgment of grants, issue a \IEEEoverridecommandlockouts
% after \documentclass

% for over three affiliations, or if they all won't fit within the width
% of the page, use this alternative format:
% 
%\author{\IEEEauthorblockN{Michael Shell\IEEEauthorrefmark{1},
%Homer Simpson\IEEEauthorrefmark{2},
%James Kirk\IEEEauthorrefmark{3}, 
%Montgomery Scott\IEEEauthorrefmark{3} and
%Eldon Tyrell\IEEEauthorrefmark{4}}
%\IEEEauthorblockA{\IEEEauthorrefmark{1}School of Electrical and Computer Engineering\\
%Georgia Institute of Technology,
%Atlanta, Georgia 30332--0250\\ Email: see http://www.michaelshell.org/contact.html}
%\IEEEauthorblockA{\IEEEauthorrefmark{2}Twentieth Century Fox, Springfield, USA\\
%Email: homer@thesimpsons.com}
%\IEEEauthorblockA{\IEEEauthorrefmark{3}Starfleet Academy, San Francisco, California 96678-2391\\
%Telephone: (800) 555--1212, Fax: (888) 555--1212}
%\IEEEauthorblockA{\IEEEauthorrefmark{4}Tyrell Inc., 123 Replicant Street, Los Angeles, California 90210--4321}}




% use for special paper notices
%\IEEEspecialpapernotice{(Invited Paper)}




% make the title area
\maketitle


\begin{abstract}
%\boldmath
The population of elderly citizens forms one of the largest groups of people in many developed countries across the world. Research in the past has focused on the usability concerns of the elderly with traditional PCs and laptops. This paper explores how recent research available in literature might mitigate some of those concerns through the use of touch screen interfaces of modern tablets and mobile phones. A simple usability test to validate such assertions is also proposed.


\end{abstract}
% IEEEtran.cls defaults to using nonbold math in the Abstract.
% This preserves the distinction between vectors and scalars. However,
% if the conference you are submitting to favors bold math in the abstract,
% then you can use LaTeX's standard command \boldmath at the very start
% of the abstract to achieve this. Many IEEE journals/conferences frown on
% math in the abstract anyway.

% no keywords




% For peer review papers, you can put extra information on the cover
% page as needed:
% \ifCLASSOPTIONpeerreview
% \begin{center} \bfseries EDICS Category: 3-BBND \end{center}
% \fi
%
% For peerreview papers, this IEEEtran command inserts a page break and
% creates the second title. It will be ignored for other modes.
\IEEEpeerreviewmaketitle



\section{Introduction}
% no \IEEEPARstart
The older age-group is the fastest growing population in many parts of the world, growing at 1.9\% compared to the mean of 1.2\%. Currently, the number of people of age 60 or over is around 660 million and is expect to reach 2,000 million by the year 2050\cite{UN}. In developed countries, this trend is more pronounced with about 37\% of the population expected to be over the age of 60 by 2050.  With increased time for leisurely activities, need to communicate with family members and services that provide shopping, investment, travel and health information, elderly people also constitute the fastest growing demographic of internet and computer users \cite{hanson2001web}. This large target market alone provides a clear financial incentive for developers to incorporate elderly usability needs into their system design. Computer use has also shown to improve the quality of life, minimize social and psychological problems among the elderly\cite{karavidas2005effects}. The regular use of computers has been shown to increase a sense of independence, reduce social isolation and empower the lives of the elderly \cite{wolf2010elderly}. 

  Research in the past has outlined some specific areas of usability problems in the domain of computer interface design including vision, dexterity, cognition and hearing\cite{hanson2001web}. Users in the advanced age groups are also susceptible to reduction in the range of visual accommodation, a loss of contrast sensitivity, decreases in dark adaptation, declines in color sensitivity, and a heightened susceptibility to problems with glare\cite{czaja2002designing}. Elderly users are also more susceptible to depression and frustration when faced with typical error messages like "fatal error" or "illegal operation" \cite{wolf2010elderly}.  Such experiences combined with spam, pop-ups, pornography and unwanted emails result in frustration at the length of time it takes elderly users to learn computer skills \cite{gatto2008computer}. 
  
This paper attempts to elaborate on how a select few problems identified in literature could possibly be mitigated with the use of modern devices like tablets and other touch screen interfaces. 

The rest of the paper is organized as follows. Section II,III and IV elucidate some specific usability problems faced by this group, and possible solutions offered by current tablets and touch screen devices. Section V describes the metrics for a comprehensive test to evaluate the usability of tablets and other touch screen devices for the elderly. 


% You must have at least 2 lines in the paragraph with the drop letter
% (should never be an issue)


\section{Vision}

\subsection{Background:} Problems with deteriorating vision are common in the elderly population.
\begin{figure}[!h]
\centering
\includegraphics[scale=0.6]{chrome}
\caption{Google Chrome Browser in Windows 8}
\label{chrome}
\end{figure} 
 A decrease in contrast discrimination, acuity and color perception has been shown to create difficulties with font sizes, colors and images \cite{hanson2001web}. When a large number of icons are available, or when the size and meaning of the icon is not immediately apparent, elderly users have been shown to express difficulty in navigating through tasks. The use of a many icons has also been shown to increase the complexity of the system \cite{czaja1996aging}. The standard setting for desktop icons are small, and most users are not familiar with customizable settings \cite{holzinger2007some}. Consider for example a typical browser in a Windows OS environment in Figure~\ref{chrome}.




An elderly user might face problems when trying to perform the simple task of exiting this application. The contrast of the small white on blue icons for closing,minimizing and expanding the browser are not very noticeable, and the different functionalities offered by individual icons are not immediately obvious for typically inexperienced elderly users. 

\subsection{Solution:}
Consider in contrast the same browser in Android and iPhone OS environments in Figure~\ref{chrome2}.
 \begin{figure}[!h]
\centering
\includegraphics[scale=0.15]{chrome2}
\caption{Google Chrome Browser in Android and iPhone}
\label{chrome2}
\end{figure} 

In Android devices, the presence of the persistent action bar in the bottom of the browser might make it easier for elderly users to exit the application. The exit and back buttons are also very visible with the high contrast of white outline on a black background. The action bar is also consistent across all applications on this device which breeds familiarity among users, making it an easy task to learn with just a few repeat uses. Display information consistency along with recognition and incorporating previous knowledge results in increased interest and motivation \cite{holzinger2007some}. Similarly, the Apple iPhone also comes equipped with a single exit button that is integrated into the hardware, making the task of exiting the application easy to learn with just a few uses.
Enlarging text with a simple pinch out method on touch screen devices is easy to learn even for first time users.  

\subsection{Potential Pitfalls:}
While touch screen devices might be useful for the particular task of exiting a web browser, research in the past has shown that the small size of icons,text and other targets in mobile devices might prove to cause some difficulty in visibility for elderly users \cite{holzinger2007some}. As an example, consider the browsers in Figure~\ref{chrome2} and notice how navigating to a new tab or a new window might prove a challenge to an elderly user. 

\section{Dexterity}
\subsection{Background:}
Elderly users have been shown to demonstrate difficulty with using a computer mouse and a keyboard owing to arthritis, tremors or other physical problems that make mobility difficult. 
In a detailed study \cite{laursen2001performance}, it was shown that the elderly group on average performed 15\% slower than the young group in tasks at various levels of precision while also registering up to 210\% more errors for the same tasks while using a mouse(Figure~\ref{dexterity}).
 \begin{figure}[!h]
\centering
\includegraphics[scale=0.7]{dexterity}
\caption{Working speed as clicks per minute (\textbf{a}) and error rate (\textbf{b})}
\label{dexterity}
\end{figure} 


 Tasks like double clicking and dragging have been shown to pose conceptual difficulty for such users \cite{hanson2001web}. Having to switch between a mouse and the keyboard also proved to be a challenging task for such users \cite{johnson2007designing}. Elderly users are also less susceptible to tactile feedback, and even with practice demonstrate difficulty in tracking objects on a screen with a mouse\cite{holzinger2007some}.

\subsection{Solution:}
A study \cite{1167982} using 45 subjects divided into three age groups showed that the pointing time with a PC mouse for the elderly was significantly longer than the other age groups. Simultaneously however, there was no observed difference in the pointing time by age group while using a touch panel (Figure ~\ref{dexterity2}).

 \begin{figure}[!h]
\centering
\includegraphics[scale=0.7]{dexterity2}
\caption{Pointing time for each group}
\label{dexterity2}
\end{figure} 

The results indicate that the elderly group finds the touch panel more useful than the PC mouse, as demonstrated by the marked improvement in performance (Figure ~\ref{dexterity3})

 \begin{figure}[!h]
\centering
\includegraphics[scale=0.7]{dexterity3}
\caption{Difference in pointing time between PC mouse and touch panel)}
\label{dexterity3}
\end{figure}

There was also no observed difference in error rates by age group, with either the touch panel or the PC mouse (Figure~\ref{dexterity2.5}). The study concludes that these results indicate that a touch panel would be desirable as in input device as it "prevents the influence of age". 
 \begin{figure}[!h]
\centering
\includegraphics[scale=0.7]{dexterity25}
\caption{Error rate for each group}
\label{dexterity2.5}
\end{figure} 

\subsection{Potential Pitfalls:}
Impaired motor skills might make it difficult for elderly users to use and operate smaller modern touch screen devices. In tests, elderly users susceptible to shaking hands have been shown to face considerable challenges operating phones for even simple operations like making a phone call. Keypads of modern devices are very touch sensitive, resulting in increased difficulty faced by users when inputting text into their phones and tablets \cite{haikio2007touch}. A detailed usability test has to be conducted to evaluate the potential benefits of using touchscreen devices for a typical elderly user. 
 
\section{Cognitive Impairment and Preconceptions:}
\subsection{Background:}
Elderly users have difficulty learning unfamiliar domains, forming cognitive models of applications, attention problems, memory impairments and lack of experience in prior computer use \cite{hanson2001web}. Research has also shown that the inability to use the technology tools leads to intense frustration among elderly users leading to feelings of nervousness around computers and mistrust of others using computers. Seniors also showed a fear concerning security issues of using a complicated and confusing machine \cite{wolf2010elderly}. 

\subsection{Solution:}
Elderly users with prior use of mobile phones study were found to be more be more willing to adopt a touch phone application. Interestingly enough, users with no prior experience with mobile phones were able to learn, adopt and use the touch interface as fluently as expert mobile users \cite{haikio2007touch}. This suggests that touch screen tablets and phones are an intuitive and natural way to select objects for elderly users, regardless of prior experience, which makes the technology easy to adapt to and can help overcome the unwillingness of such users to adapt to older PC technology.

\subsection{Potential Pitfalls:}
Elderly users may face some difficulty in performing some advanced tasks on a mobile phone, like changing batteries or recharging devices \cite{haikio2007touch}. Some elderly users also are also afraid of dropping and damaging touch screen devices, which discourages them from actively adopting the technology \cite{holzinger2007some}. Fears about security, spam, pornography and information theft might not be mitigated simply by switching over to touch devices. Mobile and tablet technology might also dissuade some adult users from having to learn and adapt to also the rapid pace at which such technologies are typically upgraded. 

\section{Usability Testing Metrics}

Previous sections of the paper showed some of the problems faced by elderly people with computer interfaces and how research in literature supports my claim that touch screen devices can be used to mitigate some of those problems. The usability tests outlined in this section will be performed on elderly user test subjects to fully support my claim. The usability tests will be designed to evaluate the performance of users on tasks typical of computer usage by the elderly \cite{white1999surfing} (Figure~\ref{usage})  

 \begin{figure}[!h]
\centering
\includegraphics[scale=0.55]{usage}
\caption{Computer usage by the elderly}
\label{usage}
\end{figure}

\subsection{Participants:}
Thirty test subjects between the ages of 60 and 75 will be chosen for the test. Participants will include members from both the genders, with a good mix of users with and without prior computer use, with and without prior mobile use and with no specific software skills or proficiency. Participants will be members with health typical of that particular age group, with no extenuating illnesses or disabilities. Studies with test users of similar characteristics have been accepted for publication in the past \cite{kobayashi2011elderly}.

\subsection{Apparatus:}
All subjects will be given an iPad tablet (9.7 inch multi touch screen  with $768\times1024$ pixel resolution), and an iPhone (3.5-inch multi touch screen with  $640\times960$ pixel resolution). Users will also share access to a standard Dell PC running Windows XP with standard input hardware (15 inch desktop monitor $1024\times768$ pixel resolution). Users will be urged to hold their personal devices in a position comfortable to them, and the shared computer will have a standard office desk and chair.

\subsection{Tasks:}
All participants will use the respective native apps to carry out the following tasks. A helper will initially demonstrate how to perform individual tasks. The duration for the testing phase will last a whole month. 

\begin{itemize}
  \item \textbf{Email} Participants will send and respond to 2 emails every week. The length of the email will be less than 250 words, which is typical for personal and professional emails.
  \item \textbf{Internet} Participants will be urged to look up sports, weather and health information on the internet typically a few times every day.
  \item \textbf{Word Processing} Participants will compose a text document of about 250 words every day. The focus of this section will be on their ability to store and retrieve stored documents from memory, and resume the word processing document.
\end{itemize}

\subsection{Metrics:}
A study \cite{holzinger2008investigating} that worked together with the elderly end users to approach the evaluation of technology from a User-Centered design perspective defined useful metrics, some of which will be replicated in this study:

\begin{itemize}
  \item How can technology be made more \textbf{trustworthy?}.
  \item How can improve \textbf{understanding} of functionality?
  \item How can technology inspire \textbf{confidence} of users abilities?
  \item How can we make the user feel in \textbf{control} of their environment?
  \item How can we explain the \textbf{benefits} of this technology to the user?
\end{itemize}

Feedback from users will be recorded as statements and then mapped to fit the above metrics.

\subsection{Hypotheses:}

This study will test the hypothesis enumerated in earlier sections and other research\cite{kobayashi2011elderly} available in literature:

\begin{enumerate}
  \item Basic touchscreen operations will be easy to learn for the elderly with just minimum training.
  \item Learning touchscreen operations will improve at a faster rate than standard PC operations, irrespective of prior knowledge of usage.
  \item Devices with larger screens (iPad) are easier for operations owing to better visibility but devices with smaller screens (iPhone) are more comfortable to use owing to easier mobility.
  \item Users will be eager to experiment with greater functionality on touch screen devices than traditional PCs.  
  \item Users will continue to face problems with unclear instructions and unclear indicators of current state will be similar across all interfaces.
  
\end{enumerate}

% An example of a floating figure using the graphicx package.
% Note that \label must occur AFTER (or within) \caption.
% For figures, \caption should occur after the \includegraphics.
% Note that IEEEtran v1.7 and later has special internal code that
% is designed to preserve the operation of \label within \caption
% even when the captionsoff option is in effect. However, because
% of issues like this, it may be the safest practice to put all your
% \label just after \caption rather than within \caption{}.
%
% Reminder: the "draftcls" or "draftclsnofoot", not "draft", class
% option should be used if it is desired that the figures are to be
% displayed while in draft mode.
%




% Note that IEEE typically puts floats only at the top, even when this
% results in a large percentage of a column being occupied by floats.


% An example of a double column floating figure using two subfigures.
% (The subfig.sty package must be loaded for this to work.)
% The subfigure \label commands are set within each subfloat command, the
% \label for the overall figure must come after \caption.
% \hfil must be used as a separator to get equal spacing.
% The subfigure.sty package works much the same way, except \subfigure is
% used instead of \subfloat.
%
%\begin{figure*}[!t]
%\centerline{\subfloat[Case I]\includegraphics[width=2.5in]{subfigcase1}%
%\label{fig_first_case}}
%\hfil
%\subfloat[Case II]{\includegraphics[width=2.5in]{subfigcase2}%
%\label{fig_second_case}}}
%\caption{Simulation results}
%\label{fig_sim}
%\end{figure*}
%
% Note that often IEEE papers with subfigures do not employ subfigure
% captions (using the optional argument to \subfloat), but instead will
% reference/describe all of them (a), (b), etc., within the main caption.


% An example of a floating table. Note that, for IEEE style tables, the 
% \caption command should come BEFORE the table. Table text will default to
% \footnotesize as IEEE normally uses this smaller font for tables.
% The \label must come after \caption as always.
%
%\begin{table}[!t]
%% increase table row spacing, adjust to taste
%\renewcommand{\arraystretch}{1.3}
% if using array.sty, it might be a good idea to tweak the value of
% \extrarowheight as needed to properly center the text within the cells
%\caption{An Example of a Table}
%\label{table_example}
%\centering
%% Some packages, such as MDW tools, offer better commands for making tables
%% than the plain LaTeX2e tabular which is used here.
%\begin{tabular}{|c||c|}
%\hline
%One & Two\\
%\hline
%Three & Four\\
%\hline
%\end{tabular}
%\end{table}


% Note that IEEE does not put floats in the very first column - or typically
% anywhere on the first page for that matter. Also, in-text middle ("here")
% positioning is not used. Most IEEE journals/conferences use top floats
% exclusively. Note that, LaTeX2e, unlike IEEE journals/conferences, places
% footnotes above bottom floats. This can be corrected via the \fnbelowfloat
% command of the stfloats package.



\section{Conclusion and Future Work}
This paper introduces the need for developers to account for the usability challenges faced by the elderly. This paper also describes some typical challenges faced by older users of traditional computer systems and theorizes how research available in literature shows how some of those problems might be mitigated through the use of touch screen devices. Finally, an usability study to test for the validity of those theoretical assertions was proposed.

Future work might explore problems elderly users face as a result of other age related disabilities like loss of hearing, use a larger test group with varying degrees of prior exposure to touchscreen use, test for more typical tasks like games or filling in online forms and making online financial transactions. Future work may also expand the type of devices to use Android and other mobile operating systems, and a future study may also include a younger test group for reference.




% conference papers do not normally have an appendix


% use section* for acknowledgement
\section*{Acknowledgment}


The author would like to thank the instructor Dr.Pastel for giving him the opportunity to learn about challenges faced by the elderly, and thereby view his own grandmother with more compassion and understanding when summoned upon to 'fix' her computer.





% trigger a \newpage just before the given reference
% number - used to balance the columns on the last page
% adjust value as needed - may need to be readjusted if
% the document is modified later
%\IEEEtriggeratref{8}
% The "triggered" command can be changed if desired:
%\IEEEtriggercmd{\enlargethispage{-5in}}

% references section

% can use a bibliography generated by BibTeX as a .bbl file
% BibTeX documentation can be easily obtained at:
% http://www.ctan.org/tex-archive/biblio/bibtex/contrib/doc/
% The IEEEtran BibTeX style support page is at:
% http://www.michaelshell.org/tex/ieeetran/bibtex/
\bibliographystyle{IEEEtran}
% argument is your BibTeX string definitions and bibliography database(s)
\bibliography{FinalTopic}
%
% <OR> manually copy in the resultant .bbl file
% set second argument of \begin to the number of references
% (used to reserve space for the reference number labels box)







% that's all folks
\end{document}


